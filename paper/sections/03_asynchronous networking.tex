% !TeX root = ../main.tex
% Add the above to each chapter to make compiling the PDF easier in some editors.

\section{Asynchronous networking using Tokio} \label{asynchronous-networking}
In chapter \ref{networking-in-rust}, basic examples for TCP and UDP communication in Rust were shown. In these programs
the current thread was blocked when waiting for new connections or incoming messages. This not only results in the
program having to wait for a new message to continue executing the code, but it also means that it is necessary to
create a new thread for each connection if the server should listen to multiple connections at once. This principle is
called synchronous programming. In contrast to that, there are no blocking functions in asynchronous programming.
Instead, results are wrapped in so-called futures which represent a result of an otherwise blocking function. The
result can be unwrapped when it is needed instead of waiting for the execution in advance. \cite{async-rust}

\subsection{The Tokio crate}
The core library of Rust is kept very small. However, the package manager \textit{Cargo} allows an easy import of
third-party packages. This way these extensions are not bound to the release cycle of Rust and breaking changes will
only affect one single package instead of the whole library. For compatibility reasons, a project could keep using an
older version of a package while still upgrading the standard library. The packages of Rust are called crates.

A crate is similar to a binary or library. By adding the name and the version to the dependencies of a project, it gets
imported automatically. Some crates also offer feature flags to decide which parts of the library should be included.
This prevents the whole library from getting compiled with the code.

To introduce asynchronous programming to Rust it is feasible to use a crate that implements a runtime for the
execution of asynchronous functions. Rust already provides a minimal implementation for async code, but a lot of
key parts like executors, tasks, and reactors are missing. Crates like \textit{Tokio} build on top of these basic
implementations, filling in the gaps of necessary parts which are not implemented yet. As the documentation
\cite{tokio-doc} states, \textit{Tokio} is a runtime for writing reliable network applications. It offers tools for
working with asynchronous tasks as well as APIs for typical blocking operations such as sockets and filesystem
operations.

A task in the context of a \textit{Tokio} program is a non-blocking piece of code. The \rust{tokio::task::spawn} method
can be used to schedule the task on the \textit{Tokio} runtime and use \rust{await} to wait for the returned value.
This is a typical behavior in asynchronous programming. A task can be compared to a promise in the programming language
JavaScript.

The \textit{Tokio} runtime consists of three major parts: An I/O event loop, which handles I/O events and dispatches
them to the waiting tasks, a timer to run tasks after a certain period of time and a scheduler, which executes the
tasks on the different threads. One way to enable the \textit{Tokio} runtime is to mark the main function with the
\rust{async} keyword and add the \mintinline{rust}{#[tokio::main]} attribute macro above.

The scheduler swaps around all tasks which need to be run. This means there might be more tasks than threads at the
same time whilst still allowing concurrent execution of all tasks. To tell the scheduler that the program is currently
waiting for another task to finish before it can continue running, one can call \rust{await} on the method. This
effectively stops the execution of the current task, waits for the other task to finish, and then unwraps the returned
result. Contrary to an execution without the \textit{Tokio} runtime, the scheduler knows when the current task can
continue running and uses the otherwise unused CPU time by letting another task run instead.

The following chapters are going to use the crate \textit{Tokio} mentioned before as a runtime for all the asynchronous
code provided in the examples.

\subsection{Programming with futures}
Let us now take a look at the small demonstration of futures in \textcolor{teal}{Figure \ref{future}}.

\begin{figure}[ht]
    \begin{minted}{Rust}
async fn async_function() {
    println!("Started task1");
    sleep(Duration::from_secs(5))
        .await;
    println!("Finished task1");
}

#[tokio::main]
async fn main() {
    let future1 = async_function();

    let future2 = async {
        println!("Started task2");
        sleep(Duration::from_secs(3))
            .await;
        println!("Finished task2");
    };

    '\textcolor{macro}{join!}'(future1, future2);
}
    \end{minted}
    \caption{Future demonstration}
    \label{future}
\end{figure}

The declared function types play an important role when working with \textit{Tokio}. This is why the example of
\textcolor{teal}{Figure \ref{future}} also includes the entire function definitions.

The main function is marked as \rust{async} and the \textit{Tokio} macro is used to create a runtime for our program to
run in. In \textcolor{orange}{line 10}, \rust{async_function} is called. The value returned by the function has the
type \rust{std::future::Future}, which wraps the result of the computation that is going to get evaluated later. The
\rust{Future} is stored in the variable \rust{future1}. This way the \textit{Tokio} runtime decides when to evaluate
the function. If \rust{async_function} was not marked with \rust{async}, the main thread executing the program would
enter the function, completely evaluate it and after that continue executing the main function.

The type of \rust{future2} is also a \rust{Future}, the only difference is the use of an \rust{async} block instead of
an extra function to define the asynchronous code.

In the end, the macro \nohighlight{join!} is used to wait for both futures to be evaluated completely while executing
them concurrently.

Both functions first print a short message, signaling that they have started executing, then sleep a certain amount of
time using \rust{tokio::time::sleep} and then print that they have stopped executing. The key part of this is, that the
function wrapped by \rust{future1} is called first and sleeps for five seconds, the function wrapped by \rust{future2}
is called afterward and only sleeps for three seconds.

In a blocking environment, this would result in the program taking more than eight seconds to execute, because the
returned value of \rust{future1} is evaluated first and only after that the value of \rust{future2}. Therefore, the
output would look like this:

\begin{minted}[frame=lines]{TeX}
Started task1
Finished task1
Started task2
Finished task2
\end{minted}

The only way to avoid the synchronous execution would be to start two threads explicitly and execute both tasks
concurrently. If multiple tasks have to run in parallel and include a lot of blocking operations, it would cause a lot
of overhead to the program if a thread for each task was created.

On the other hand in an asynchronous environment, the scheduler stops the execution of the current task and swaps in
another task which is currently waiting for execution as soon as the program reaches an \rust{await} point. This way,
the sleep function does not block the entire program flow and the function wrapped by \rust{future2} finishes first,
even when only executing on one thread. The resulting output of the program looks like this:

\begin{minted}[frame=lines]{TeX}
Started task1
Started task2
Finished task2
Finished task1
\end{minted}

\subsection{Asynchronous TCP communication}
\textit{Tokio} offers the module \rust{tokio::net}, which contains a TCP and UDP network API similar to the one of the
standard library used in chapter \ref{networking-in-rust}. These functions are marked as \rust{async}, blending in with
the rest of the \textit{Tokio} ecosystem.

Let us first use \textit{Tokio} to spawn a new task that handles a connection by reading all incoming messages and
printing the output to the command line. The code in \textcolor{teal}{Figure \ref{async-tcp-client-handling}}
implements basic asynchronous client handling.

\begin{figure}[ht]
    \begin{minted}{Rust}
tokio::spawn(async move {
    let mut reader =*BufReader::new(&mut stream);

    loop {
        let mut buf = String::new();

        match reader
            .read_line(&mut buf).await {
            Ok(0) => break,
            Ok(size) =>*print!("{}", buf),
            Err(e) => {
                eprintln!("{}", e);
                stream
                    .shutdown().await?;
                break;
            }
        }
    }

    Ok::<(), Error>(())
});
    \end{minted}
    \caption{Asynchronous TCP client handling}
    \label{async-tcp-client-handling}
\end{figure}

The current connection is represented by the variable \rust{stream}, which has the type \rust{tokio::net::TcpStream}.
The function \rust{tokio::spawn} takes a \rust{Future} as an argument and creates a new asynchronous task. When and how
this task gets executed depends on the configuration of the current runtime.

The function is given an asynchronous code block as an argument, which returns a \rust{Future} as seen in Figure
\textcolor{teal}{\ref{future}}. Here additionally to the \rust{async} keyword, \rust{move} is used. This means the
asynchronous block will take ownership of all variables referenced within it. Without this keyword, all variables would
be bound to the scope of the code surrounding it. This is especially important as our asynchronous code needs full
ownership of the \rust{stream} variable in order to read from it and shut it down in case of an error.

The rest of the code works similar to the synchronous TCP client handling of
\textcolor{teal}{Figure \ref{tcp-client-handling}}: In \textcolor{orange}{line 2} a buffered reader instance is
created for more efficient reading from the \rust{TcpStream}. After that, the program enters a loop to read data from
the connection repeatedly. In \textcolor{orange}{line 5} a mutable \rust{String} named \rust{buf} is created to save
the incoming data. Then the method \rust{read_line} of \rust{reader} is called with \rust{buf} as an argument. As the
reader instance has the type \rust{tokio::io::BufReader} in this asynchronous implementation, \rust{read_line} does not
return a simple \rust{Result} but rather a \rust{Future} which wraps the returned value. This means we can use an
\rust{await} point to tell the scheduler it can stop running the current program until new data has arrived.

If reading is successful and no bytes were read, it means that the connection was closed and we can break out of the
loop. If at least one byte was read, the program enters the second match arm and the read data gets printed to the
command line.

If reading has failed, the program enters the third match arm and prints the error to the command line. After that, it
tries to shut down the stream. Here the \rust{shutdown} method writes all buffered data to the stream and shuts down
the write instance of the stream using a system call. Different from the \rust{shutdown} method of
\rust{std::io::TcpStream}, it does not take an enum as an argument to control how the stream should be shut down. The
method also returns a \rust{Result} wrapped by a \rust{Future}. First, the \rust{Future} gets unwrapped by calling
\rust{await} on it. Afterward, the \rust{Result} is unwrapped using the \rust{?} operator and the program breaks out of
the loop.

At the end of the task, we have to specify the type of the \rust{Result} returned explicitly. The reason for this is
the use of the \rust{?} operator in \textcolor{orange}{line 14} to unwrap the returned value of the \rust{shutdown}
method. As \rust{async} blocks do not have an explicitly specified return type, the compiler sometimes fails to infer
the error type of the async block. This will trigger the compiler error \nohighlight{type annotations needed}. By
specifying the exact type of the \rust{Ok} enum variant, it is possible for the compiler to infer the return type and
we are able to use the \rust{?} operator. Another way to handle this would be to use the \rust{expect} method or
creating an extra asynchronous function instead of using \rust{async} blocks. \cite{async-rust}

As in chapter \ref{networking-in-rust} we are now able to wrap our code from
\textcolor{teal}{Figure \ref{async-tcp-client-handling}} into a function called \rust{handle_client} to be able to
call it in \textcolor{teal}{Figure \ref{async-tcp-listener}}. Next, we are going to open the TCP streams and call
\rust{handle_client} with the stream as an argument. As mentioned before, having TCP connections to multiple clients
would require a thread for each connection. \textit{Tokio} solves this problem as the event loop manages and listens to
every connection. Instead of creating a thread for each connection, it is only needed to spawn a task for each TCP
stream. This is why the code example in Figure \textcolor{teal}{\ref{async-tcp-listener}} supports connecting to
multiple clients.

\begin{figure}[ht]
    \begin{minted}{Rust}
let listener = TcpListener::bind(ADDRESS).await?;

loop {
    let (stream, _) = listener.accept().await?;

    handle_client(stream);
}
    \end{minted}
    \caption{Asynchronous TCP listener}
    \label{async-tcp-listener}
\end{figure}

First, a \rust{tokio::net::TcpListener} is created by binding it to the address saved in the \rust{&str} constant
\rust{ADDRESS}. The value corresponds to the IP address and port the server should listen to. The future result is
then unwrapped using \rust{await} as well as \rust{?} and saved to the variable \rust{listener}. As the
\rust{TcpListener} of \textit{Tokio} does not implement the \rust{incoming} method, we have to call the \rust{accept}
method of the listener ourselves. This method returns a tuple including the \rust{TcpStream} and the address of the new
connection, wrapped by a \rust{Result} to handle errors and a \rust{Future} to make the waiting operation non-blocking.
After that, the received stream can be given to our \rust{handle_client} function, which spawns the new task handling
the client. As the client handling runs in an extra task, the program can continue to run and accept new connections
without having to wait for the current one to disconnect.

To connect to this server one could just use the TCP client of \textcolor{teal}{Figure \ref{tcp-client}}, as both
servers use the same protocol and can work together interchangeably. To showcase how the \textit{Tokio} library affects
a client implementation, there is a smaller demonstration of an asynchronous TCP client in
\textcolor{teal}{Figure \ref{async-tcp-client}}.

\begin{figure}[ht]
    \begin{minted}{Rust}
let mut stream = TcpStream::connect(ADDRESS).await?;

let mut reader = BufReader::new(stdin());

loop {
    let mut buf = String::new();

    match reader
        .read_line(&mut buf).await {
        Ok(_) => {
            let bytes = buf.as_bytes();

            stream
                .write_all(bytes)
                .await?;
        }
        Err(e) => {
            eprintln!("{}", e);
            stream.shutdown().await?;
            break;
        }
    }
}
    \end{minted}
    \caption{Asynchronous TCP client}
    \label{async-tcp-client}
\end{figure}

In the beginning a \rust{tokio::net::TcpStream} instance is created by connecting to \rust{ADDRESS}, a constant in
which the address of the server is stored as a \rust{&str}. Different to the synchronous implementation, we are not
going to read from the command line directly but rather wrap it using a \rust{tokio::io::BufReader} instance in order
to read from it without blocking the task.

The loop for reading messages on the command line and sending them to the server is also very similar to the
synchronous implementation in \textcolor{teal}{Figure \ref{tcp-client}}: In \textcolor{orange}{line 6} a new
\rust{String} called \rust{buf} is created to store the input of the user. Then the \rust{read_line} method with
\rust{buf} as an argument is invoked. As this is an asynchronous method, we can use the \rust{await} keyword to give up
the computing resources until new input is written into the buffer. After new input has arrived, the scheduler
continues to execute the current task and the return value of \rust{read_line} is matched.

If reading is successful, the program enters the first match arm. In \textcolor{teal}{Figure \ref{tcp-client}} this
match arm included a check whether the read \rust{String} is equal to "exit" to offer a way to close the connection
other than just by forcing it. This part is omitted in \textcolor{teal}{Figure \ref{async-tcp-client}} as it works in
the same way. The input is converted to bytes using \rust{as_bytes} and saved in the variable \rust{bytes}. After that,
the byte slice is given as an argument to the \rust{write_all} method of \rust{stream}. The returned \rust{Future} is
again unwrapped using \rust{await} and the contained \rust{Result} unwrapped using \rust{?}.

If reading has failed, the error is printed to the error output, the program tries to shut down the stream using its
\rust{shutdown} method, unwrap the \rust{Future} and \rust{Result} using \rust{await} and \rust{?} and then break out
of the loop.

When running the TCP client, the asynchronous client does not differ as much from the synchronous client as the server
implementations differ from each other. The server implementation can use the scheduling from \textit{Tokio} to handle
multiple connections at once. The client on the other hand is idle while waiting for input on the command line anyway
because it does not have any additional tasks. When expanding the implementation to a complete chat client which can
both send and receive messages, a second task would be needed to listen for new data on the connection. Here an
asynchronous environment would be more useful.

\subsection{Asynchronous UDP communication}
Concerning asynchronous UDP communication, \textit{Tokio} offers the \rust{struct} \rust{tokio::net::UdpSocket} which
works similar to the \rust{UdpSocket} of the standard library used in \textcolor{teal}{Figures \ref{udp-receiver}} and
\textcolor{teal}{\ref{udp-sender}} for the UDP sender and receiver.

Porting the synchronous UDP implementation to an asynchronous one works the same as we have done for the TCP client and
server in the previous subchapter: Blocking operations are replaced by non-blocking operations returning futures. If
the current task has to wait for a \rust{Future} to evaluate, \rust{await} can be used.

As UDP does not implement persistent connections between the sender and the receiver and only binds to an address, the
behavior of the asynchronous UDP implementation will also not differ as much from the synchronous example, similar to
\textcolor{teal}{Figure \ref{async-tcp-client}}. At least this is the case when the sender and receiver are as strictly
separated from each other as they are in the synchronous UDP example. Otherwise, the \rust{struct}
\rust{std::sync::Arc} could be used to share the ownership of the \rust{UdpSocket} to have a sending and receiving task
in the same client. This is possible because both the \rust{send_to} and \rust{recv_from} methods of the socket do not
require a mutable reference to the socket. If this was not the case, it would not be possible to share one socket for
two tasks as there can only be one mutable reference to a certain piece of data in Rust.
