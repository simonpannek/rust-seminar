% !TeX root = ../main.tex
% Add the above to each chapter to make compiling the PDF easier in some editors.

\section{Introduction} \label{introduction}
Many modern devices are using the internet to communicate with each other. This concept of connecting to each other
dates back to the 1960s when the protocols for one of the first computer networks, the ARPANET, were developed. It
allowed large existing host computers with different configurations to communicate with each other, even allowing
indirect package transmission by using the hosts as a bridge between two not directly connected hosts \cite{f70arpa}.
Since then, new protocols have been introduced, additional security, for example through encryption was added and a lot
of these changes were standardized in order to let devices produced by diverse manufacturers seamlessly work together.

One advantage of standardization is that almost every popular programming language supports these frequently deployed
network protocols, whether natively in its standard library or through simple imports of third-party libraries. All
these options raise the question of what language to use for own application purposes. There is no right answer to this
question as it depends on many aspects such as the particular use case, the running environment, time, and cost of
development. This paper highlights the advantages the programming language Rust has to offer when it comes to running
it as a backend of a network application and showcases simple code examples about communication using the Transmission
Control Protocol (TCP) and User Datagram Protocol (UDP).

There are usually different approaches when getting to pick a language for networking, especially when working with a
transport protocol and a custom application protocol on top of it.

One option would be using a scripting language like Python or JavaScript. With such languages, it is possible to easily
import libraries that offer a basic network API and a simple syntax allowing fast prototyping. As these languages
usually are interpreted, they often operate much slower compared to executing compiled binary files as the instructions
first have to get translated by the interpreter before being run.

If speed and scalability are important factors (as they are most of the time), a compiled language could be a good
choice. This is the reason why many network libraries are written in C or C++. However, these two languages do not
guarantee memory safety. Mistakes by the programmer can result in memory usage errors like invalid accesses or leaks.

This might be a problem, in particular when working with network applications as it is common to read from and write to
buffers concurrently. For example, if incoming information is just written to a buffer without checking the buffer
size, an attacker might be able to overwrite data in the memory and change the control flow of the program. This is
called a buffer overflow attack and can lead to an attacker being able to control the host running the network
application. \cite{c00buffer}

Another problem with C and C++ is thread-safety. Parallelization and preventing race conditions are completely up to
the programmer. When sharing data between multiple threads without proper synchronization, many errors will only occur
when used in production with a lot of concurrent access and a higher load on the server.

Rust is an open-source programming language developed by Mozilla introducing memory-safety without a runtime or garbage
collection. Its standard library provides networking primitives for basic TCP/UDP communication, enabling socket
communication and error handling out of the box. \cite{rust-language}

Rust is also a compiled language, but the compiler guarantees memory- and thread-safety. This means it profits from the
fast speed compiled languages have and also prevents most of the vulnerabilities mentioned above from happening, thanks
to Rust rejecting to compile code that could introduce such issues.

Network operations are always prone to failure because the success of such an operation depends on outside factors like
other hosts and a stable network connection. Rust cannot stop these errors from happening in advance as it has to rely
on system calls of the operating system. Instead, it wraps the returned information of such unsafe calls into enums of
the type \rust{std::result::Result}. This enum forces the programmer to handle errors in some kind of way as it is
required to unwrap the \rust{Result} before being able to access the returned value, preventing the program to continue
running in a faulty state.

Another selling point of Rust is its well-integrated package manager \textit{Cargo}, making it possible to easily
import, update and integrate third-party libraries called crates and ship them with an own application.

To see, how these benefits apply in basic TCP and UDP communication, chapter \ref{networking-in-rust} will first give a
short overview of ownership, borrowing and lifetimes. After that, it will explain the implementations of a TCP client
and server as well as a UDP sender and receiver, each programmed in Rust using the standard library. Chapter
\ref{asynchronous-networking} is going to explain, why the implementations of the previous chapter are not always
optimal as the standard library uses blocking operations for network calls. As a solution, asynchronous programming
using a library called \textit{Tokio} and the principle of futures are introduced. After that, an asynchronous server
and client code example are shown to explain possible differences and advantages compared to a synchronous and blocking
approach. Chapter \ref{application-layer-communication} leaves the transport layer where protocols like TCP and UDP are
explicitly used and is going to shift the focus to higher-level communication on the application layer. In the end,
the discussed results will get summarized.
