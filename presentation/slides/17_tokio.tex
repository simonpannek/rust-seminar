% !TeX root = ../main.tex
\begin{frame}{Asynchronous Networking in Rust}
    \textbf{The Tokio crate}

    \begin{itemize}
        \item<2-> Tokio is a runtime for writing reliable network applications \cite{tokio-doc}
        \item<3-> Tokio consists of three major parts:
        \begin{enumerate}
            \item<4-> I/O event loop
            \item<5-> Timer
            \item<6> Scheduler
        \end{enumerate}
    \end{itemize}

    \enote{
        \item Easily imported and managed via Cargo
        \item Includes feature flags \so specify which parts of the library should be included
        \item Parts of tokio:
        \begin{enumerate}
            \item I/O event loop: Handles I/O events and dispatches them to the tasks waiting for them
            \item Timer: Runs tasks after a certain period of time
            \item Scheduler: Executes the tasks on the different threads
        \end{enumerate}
        \item Enables the use of \rust{async}, \rust{await} and other features of asynchronous Rust
        \item \rust{tokio::net} includes networking functions similar to \rust{std::net}
    }
\end{frame}
