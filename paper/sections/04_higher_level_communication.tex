% !TeX root = ../main.tex
% Add the above to each chapter to make compiling the PDF easier in some editors.

\section{Application Layer communication} \label{application-layer-communication}
In chapters \ref{networking-in-rust} and \ref{asynchronous-networking}, transport layer protocols were used to
explicitly send text messages formatted as byte slices from a client to a server. This may be enough for simple
applications like a chat server, but in many cases, another abstraction on top of that is needed to transfer special
data like a \rust{struct} through a connection. This is where protocols belonging to the application layer are
applicable.

The following chapter will give a small insight into the principle of how asynchronous communication using
\textit{Tokio} can also be used together with higher-level protocols. In particular the Hypertext Transfer Protocol
(HTTP), a protocol used for data transmission for the World Wide Web is going to be utilized in order to send a
\rust{struct} from a client to the server and receive a response as a confirmation. HTTP is usually based on the TCP
protocol but can also be implemented using other transport protocols, which offer a confirmation, if sending data was
successful.

\subsection{Serialization}
Before sending anything over HTTP, a piece of data to send is needed. For this, we are going to introduce the
\rust{struct} called \rust{Data} in Figure \ref{data-struct}, which will both be implemented in the HTTP client and
server.

\begin{figure}[ht]
    \begin{minted}{Rust}
#[derive(Serialize, Deserialize, Debug)]
struct Data {
    number: u32,
    boolean: bool,
}
    \end{minted}
    \caption{Data struct}
    \label{data-struct}
\end{figure}

HTTP still is a text-based protocol. This means before being able to send the \rust{struct} from the client to the
server, we have to transform it from a \rust{struct} to a \rust{String} and back, using a principle called
serialization and deserialization.

For this, we are going to use a crate called \textit{Serde}. \textit{Serde} allows us to let our \rust{struct} derive
the traits \rust{Serialize} and \rust{Deserialize}. This way the compiler is capable of providing a basic
implementation to convert \rust{Data} to a JSON \rust{String} and the other way around. JSON stands for JavaScript
Object Notation and is a standard file format typically used for this exact purpose.

We additionally derive \rust{Debug} to implement the trait \rust{std::fmt::Debug}. This serves the purpose of printing
the \rust{struct} in a human-readable format, typically used for debugging purposes.

\subsection{HTTP communication}
Now that we can convert the \rust{struct} into a format that can be sent over HTTP, next, a server is needed to accept
incoming requests. There are multiple crates that enable accepting and handling HTTP requests in Rust. For our HTTP
server implementation, we are going to use a crate called \textit{Warp}.

\textit{Warp} is a simple web server framework, which builds on top of the crate Hyper, an HTTP implementation for Rust
\cite{warp-doc}. \textit{Warp} makes use of asynchronous programming and seamlessly works together with \textit{Tokio}.
An important concept of \textit{Warp} is the so-called filter: \rust{warp::Filter} is a \rust{struct} that extracts
data from a request, handles the data, and sends a response back to the sender of the request. The \rust{Filter}
implements two important methods:

First the method \rust{and}, which takes another \rust{Filter} as an argument and returns
them as a new, combined \rust{Filter}. This way one can chain together multiple \rust{Filter} instances and handle
requests more precisely.

Secondly, the method \rust{map} takes a function as an argument. This function receives the extracted data from the
request and maps it to another value. In return, the value gets sent back to the client sending the request.

\rust{Filter} also implements more methods useful for rejecting certain requests or customizing filters to take
different sets of arguments in a request, but these methods will not be needed for our example.

In \textcolor{teal}{Figure \ref{http-server}} there is a small example implementation of a web server written using
\textit{Warp}.

\begin{figure}[ht]
    \begin{minted}{Rust}
let filter = warp::'\textcolor{macro}{path!}'("data")
    .and(warp::post())
    .and(warp::body::json())
    .map(|data: Data|*format!("Received: {:?}", data));

warp::serve(filter).run(ADDRESS).await;
    \end{minted}
    \caption{Warp HTTP server}
    \label{http-server}
\end{figure}

First, a \rust{Filter}, which matches a path of a possible request sent to the server, has to be created. Here the
macro \nohighlight{warp::path!} can be utilized. It returns a \rust{Filter} matching the exact path segment, for our
example we are going to send the data to the path "/data".

Next, we have to specify, what the request looks like exactly. This enables us to reject invalid requests
automatically. In this case, we only want to use a POST request, a common HTTP method and used to accept data sent in
the body of the incoming request. We can do this by creating a \rust{Filter} requiring the POST method using
\rust{warp::post()} and then combining both filters using the previously mentioned \rust{and} method.

When implementing the \rust{struct} which gets sent, we have used serialization into the JSON format. This is why the
web server should expect a body encoded in JSON and try to parse the data into a \rust{Data} instance. A \rust{Filter}
provides this functionality. It can be created using the function \rust{warp::body::json()} and is again combined using
\rust{and}.

Finally, the extracted \rust{Data} instance can be mapped to a response, which in result is sent back to the client.
Here, simply formatting it to a \rust{String} using the \nohighlight{format!} macro works fine. The \rust{{:?}}
brackets the formatting macro to use the \rust{std::fmt::Debug} trait derived by \rust{Data}.

The function \rust{warp::serve} takes the \rust{Filter} defined above as an argument and creates a new instance of the
\rust{struct} \rust{warp::Server}. This server represents our web server, filtering requests according to the filter
given at initialization.

To run this server, the \rust{struct} implements an asynchronous method \rust{run}, taking the address it should listen
to as an argument and running the \rust{Server} on the current thread until the program is closed. Here, \rust{ADDRESS}
has the type \rust{([u8, 4], u16)}, a tuple with the IP address as a byte slice with a length of 4 and the port
represented as an unsigned short.

To send the \rust{struct} to this web server, we need to create a HTTP client. \textit{Reqwest} is a crate often used
for sending simple HTTP requests to a web server and like \textit{Warp} also makes use of asynchronous methods and
requires some kind of runtime like \textit{Tokio}, at least when using its default API. In
\textcolor{teal}{Figure \ref{http-client}} there is an example implementation of a \textit{Reqwest} HTTP client sending
a custom \rust{struct} of the type \rust{Data} to the \textit{Warp} server.

\begin{figure}[ht]
    \begin{minted}{Rust}
let client = reqwest::Client::new();

let response = client
    .post(URL)
    .json(&Data {
        number: 5,
        boolean: true,
    })
    .send()
    .await
    .expect("Sending request");

println!("{}", response.text()*.await.expect("Get text"));
    \end{minted}
    \caption{Reqwest HTTP client}
    \label{http-client}
\end{figure}

In \textcolor{orange}{line 1} an instance of the type \rust{reqwest::Client} is created in order to send asynchronous
requests to the server. On this client, we call the method \rust{post}, which creates a \rust{reqwest::RequestBuilder}
representing a post request. The argument \rust{URL} is equal to the HTTP URL of the targeted web server. To give it
the data as a payload, the method \rust{json} is called with a reference to our custom \rust{Data} object. This method
uses the \rust{Serialize} trait of the \rust{struct} to convert the data into a JSON \rust{String}. Then the
\rust{send} method can be invoked to construct the request and send it to the target URL. To save the result of the
request, the returned value is first unwrapped using \rust{await} and \rust{expect}.

\rust{response} has the type \rust{reqwest::Response}. To print the resulting text answered by the server, the
asynchronous method \rust{text} can be used. The \rust{String} has to be unwrapped using \rust{await} and
\rust{expect}, similar to the returned value of \rust{send}.

If we run the example of \textcolor{teal}{Figure \ref{http-client}} with the web server of
\textcolor{teal}{Figure \ref{http-server}} running at the same time, the client prints the response
\nohighlight{Received:} \nohighlight{Data { number: 5, boolean: true }}.
